\documentclass{article}
\usepackage{amssymb}

\title{Semi-groupoid Algebra}
\author{Lgarithm}
\date{2013-11-10}

\begin{document}
% \maketitle

\section{Semi-groupoid Algebra}
Let $M$ be a semi-group, $R$ a ring, $f : M \to R$,
we write $f = \sum f_m x^m$ formally and call $f_m \in R$ the coefficient of $f$ at $x^m$,
where $f_m$ is the image of $m$ in $R$ under $f$.
This kind of symblic representation can be used in
mapping between any sets.

Let
$$R[[M]] = \{f \vert f : M \to R\}$$
the algebra of formal series,
and
$$R[M] = \{f \vert f : M \to R, \# \{m \vert f(m) = 0 \} < +\infty \}$$
then algebra of polynomials.
Notice that $R[[M]]$ is the free $R$-module with basis $M$.
Define a binary operation on $R[[M]]$
$$fg = \sum (\sum_{hk = m} f_h g_k) x^m$$

Now take $G = (\mathbb N, +)$, the semi-ring of natural numbers,
with addition operation,
then $R[G]$ is right the polynomial ring of one variable,
we usually denoted by $R[X_1]$.
The formal sum $\sum f_m x^m$ now becomes the generating function of the coefficient sequence.

When $G = \mathbb Z$, $R[G]$ becomes the Laurrent series.

When $G = (\mathbb N, \ast)$, $R[G]$ becomes the algrbra of arithmetic functions,
and the product is the Dirichlet convolution.

But the incidence algebra can not be defined in this way, which needs further genralization.
Recall the definition of incidence algebra:
Let $P$ be a poset, $Q = \{(g, h) \vert g \leq h\} \subset P \times P$,
$\alpha, \beta : Q \to R$,
$$\alpha\beta = \sum (\sum_{g\leq k \leq h} \alpha_{g,k} \beta_{k,h})x^{g,h}$$
We can define an associative partial binary operation $\circ$ on $Q$,
$$\circ : Q \times Q \to Q$$
$$(x, y) \circ (y, z) \to (x, z)$$
but it doesn't make $Q$ into a semi-group.
This kind of algebra strutcutr is called a semi-groupoid.
Howevery a semi-group is the special case of a semi-groupoid,
therefore we may generatlize the definition of semi-group algebra
to semi-groupoid algebra and still use $R[Q]$ for the incidence algebra of $P$.
For $f : P \to R$, we have another kind of product
$$(\alpha f) = \sum (\sum_{g\leq h} \alpha_{g, h} f_h)x^g$$
$$(f \alpha) = \sum (\sum_{h\leq g} f_h \alpha_{h, g})x^g$$

\section{Representation}
Now consider the representation of $R[[M]]$.
Let $V$ be a $R$-module, and $GL(V)$ the general linear algebra over $V$,
the representation of $R[[M]]$ is a homomorphism
$$R[[M]] \to GL(V)$$
this homomorphism is determined by
$$M \to GL(V)$$

\section{Linear Transform}
Since $R[[M]]$ is a free monoid, therefore each $f = \sum f_m x^m \in R[[M]]$
can be viewd as a vector in coordinate representation with basis $\{x^m\}$.
Let $g$ be a fixed element in $R[[M]]$, then
$$\circ g : f \to f \circ g$$
$$g \circ : f \to g \circ f$$
are linear transforms over $R[[M]]$ and admits a matrix form with given basis $\{x^m\}$ when $M$ is finite.
Let $M = \mathbb N$ and $g = 1 + x + x^2 + \dots = {1 \over 1 - x}$, then
$\circ g$ sends each sequence into its partial sum sequence.

Each semi-group restrict to an arbitrary subset becomes a semi-groupoid.
Let $[0, n)$ be the restriction of $(\mathbb N, +)$, then $\sum_{k < n} x^k$
is the partial sum operator for finite sequence.

\section{Generating Function}


\end{document}
