\documentclass{article}
\usepackage{amsthm}
\newtheorem{defi}{Definition}
\newtheorem{lem}{Lemma}
\newtheorem{thm}{Theorem}
\newtheorem{prof}{Proof}

\def\supp{supp}

\title{Existence of $k$-th Root of a Given Permutation}
\author{Lgarithm\\\mbox{lgaritm@gmail.com}}
\begin{document}
\maketitle

\begin{abstract}
In thie article we give the nessary and sufficient condition of
when for a given permutation $\phi$ and an integer $k$, there exists
a permutaion $\psi$ such that $\phi$ is the $k$-th power of $\psi$.
\end{abstract}

\section{Introduction}
A permutation is called a regular permutation if each cycle
in its cycle decomposition has equal length.
Any permutation can be decomposite into product of disjoint
regular permutations, the decomposite is maximum if the cycle length
in each component are distinct.

The main result is given by the following two theorems
\begin{thm}
A permutation $\phi$ is a $k$-th perfect power if and only if
each maximum regular component of $\phi$ is a $k$-th perfect power.
\end{thm}

\begin{thm}
A regular permutation of $t$ cycles, each of length $m$ is a perfect $k$-th power
if and only if for each prime factor $p$ of $(m, k)$, $p^e \vert t$,
where $p^e \vert\vert k$.
\end{thm}

In the following sections, we explain the concepts in detail.

\section{Structure of a Permutation}
\begin{defi}
Let $X$ be a finite set, a permutation $\phi$ on $X$ is a
bijection map $\phi ; X \to X$.
\end{defi}

A fixed point of a permutation is an element $x \in X$ such that
$\phi(x) = x$, and the support of $\phi$, denoted by $\supp(\phi)$
is the set of non-trival points in $X$,
that is $\supp(\phi) = \{x \vert \phi(x) \neq x\}$.
Two permutations $\phi, \psi$ are disjoint if $\supp(\phi) \cap \supp(\psi) = \emptyset$.
Disjoint permutations are commutative.

As an elementary conclusion, a permutation can be decomposite into disjoint cycles.
And such decomposition is unique if regardless of order.

The cycle decomposition of a permutation on $X$ determines a partition on $X$.
Since all partitions on a set $X$ forms a partial order $\mathcal P_X$,
we say permutation $\psi$ is finer than $\phi$ if $p(\psi) \prec p(\phi)$ in $\mathcal P_X$,
where $p(\phi)$ is the partition determined by $\phi$.
Obviously the identity permutation is finer than any other permutations,
and any permutation is finer than s single cycle.

Let $t_i$ be the length of each cycle of $\phi$,
we call the ordered sequence $t_{i_1}, \dots, t_{i_k}$ the shape of $\phi$.

\section{Regular Permutation}
\begin{defi}
A permutation is called a regular permutation if each cycle
in its cycle decomposition has equal length.
\end{defi}

\begin{lem}
$\phi^k$ is finer than $\phi$.
\end{lem}

\begin{prof}
Let $\phi = c_1 \cdots c_m$ be a cycle decomposition,
then $\phi^k = c_1^k \cdots c_m^k$ since each $c_i, c_j$ are commutative.
For a single $c_i$, of course $p(c_i^k) \prec p(c_i)$ since any permutation is
finer than a single cycle.
\end{prof}

\begin{lem}
For a cycle of length $n$, and prime $p$,
$C_n^p$ splits into $p$ cycles of length $n \over p$ when $p \vert n$,
or remains a single cycles otherwise.
\end{lem}

Let $\phi$ be a regular permutation of $k$ cycles of length $m$,
for prime $p$, $\phi$ is the $p$-th power of some permutation $\psi$
iif $p \dagger m$ or $p \vert m$ and $p \vert k$.
In the former case, $\psi$ has the same shape with $\phi$,
in the latter case, $\psi$ is consist of $k \over p$ cycles of length $mp$.

\begin{lem}
$\phi$ is a perfect $k$-th power iff each maxmium
regular component of $\phi$ is a perfect $k$-th power.
\end{lem}

\begin{prof}
The sufficiency if obvious, and the nessarity is as follows.
Assume $\phi$ is the perfect $k$-th power of $\psi$,
and $\psi$ has a regular(not nessaily maximum) decomposition $\psi = g_1 \cdots g_n$.
then $\phi$ = $g_1^k \cdots g_n^k$. Since $g_i^k$ are regular, therefore
\end{prof}

\end{document}
