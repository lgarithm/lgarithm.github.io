\documentclass{article}
% \usepackage{geometry}
% \usepackage[margin=0.5in]{geometry}



\title{Kansai no Tabi}
\date{2019.03.19 (Tue) $\sim$ 2019.03.25 (Mon)}

\begin{document}
\maketitle

\begin{abstract}
Travel to Osaka, Kyoto in the spring of 2019.
\end{abstract}

\begin{tabular}{ c | c | c | c | c }
  & date   & day  & bashuo & places of interesting \\\hline
0 & 03.19  & Tue  & Osaka & \\
1 & 03.20  & Wed  & Kyoto & \\
2 & 03.21  & Thu  & Osaka & Osakajo \\
3 & 03.22  & Fri  & Osaka & USJ \\
4 & 03.23  & Sat  & Kyoto & Inari \\
5 & 03.24  & Sun  & Kyoto & Arashiyama \\
6 & 03.25  & Mon  & - & \\
\end{tabular}

% \section*{Places of Interesting}
% \begin{tabular}{ c c c }
% \end{tabular}

\section{Arrival}
\subsection{Day 0 (2019.03.19, Tue)}
Arrive in Osaka, Kansai airport on 16:00.
Arrive at Hotel Sun White\footnote{3-7-6 Tanimachi, Osakajo} on 18:15.

\section{}
\subsection{Day 1 (2019.03.20, Wed)}
Osaka

\subsection{Day 2 (2019.03.21, Thu, National Holiday)}
Osaka
Assemble at 11:30 am.
Osakajo Koen, Osaka police station, Tsutenkaku.

\subsection{Day 3 (2019.03.22, Fri)}
Osaka, USJ

\subsection{Day 4 (2019.03.23, Sat)}
Arrive in Kyoto.
Visit Inari.
% Kiyomizudera, Sanjusangen-do,

\subsection{Day 5 (2019.03.24, Sun)}
Kyoto. Visit Arashiyama.

\section{Leaving}
\subsection{Day 6 (2019.03.25, Mon)}
Leave Kyoto at 7:00 am.
Arrive at Kansai airport on 9:45.
Back to SH 12:45 $\sim $ 14:30.

\end{document}
